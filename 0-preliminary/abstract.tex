\pagenumbering{gobble}

\begin{abstractquote}
\studentname : การสร้างกรณีทดสอบด้วยวิธีการวิเคราะห์แผนภาพการเรียกใช้งาน
(Test case generated with Call graph analysis)
\,อาจารย์ที่ปรึกษา: รองศาสตร์ ดร. ธาราทิพย์ สุวรรณศาสตร์
\end{abstractquote}

แนวคิดการออกแบบซอฟต์แวร์เชิงวัตถุนั้นถือเป็นแนวคิดการออกแบบซอฟต์แวร์ที่ช่วยให้ซอฟต์แวร์นั้นยืดหยุ่น สามารถ
นำไปปรับแก้ไขการทำงานของซอฟต์แวร์ได้ง่ายมากยิ่งขึ้น สะท้อนกับบริบทการใช้งานจริง ตลอดจนช่วยให้การบำรุงรักษา
ซอฟต์แวร์นั้นทำได้ง่ายยิ่งขึ้นเมื่อเทียบกับการออกแบบซอฟต์แวร์ด้วยแนวคิดอื่น ซึ่งแนวคิดการออกแบบเชิงวัตถุนี้เอง
จะแบ่งซอฟต์แวร์ออกเป็นส่วนย่อยตามหน้าที่และความเหมาะสมกับด้านการใช้งานของธุรกิจที่ต้องการซอฟต์แวร์
และดำเนินงานสอดประสานกันตามที่ได้ออกแบบไว้ รวมทั้งยังทำให้การปรับแก้ไขโครงสร้างของซอฟต์แวร์นั้นสะดวกมากยิ่งขึ้น
เมื่อเทียบกับการออกแบบซอฟต์แวร์โดยแนวคิดอื่น ด้วยเหตุที่มีการแบ่งซอฟต์แวร์ออกเป็นส่วนย่อยตามลักษณะงานนี้เอง
ส่งผลให้ซอฟต์แวร์ที่พัฒนานั้นจะเกิดการเรียกใช้งานคลาสหนึ่งจากอีกคลาสหนึ่ง เพื่อสื่อสารหรือส่งผ่านข้อมูลระหว่างกัน 
ตามขั้นตอนการทำงานที่กำหนดไว้ ซึ่งอาจจะเกิดข้อผิดพลาดระหว่างการเรียกหากันนี้ได้ \\ 

ซึ่งในช่วงการทดสอบการทำงานผสานกันนั้น ผู้ทดสอบจำเป็นจะต้องทราบถึงวิธีการส่งข้อมูลระหว่างคลาส
และการเรียกใช้งานซึ่งกันและกัน เพื่อนำข้อมูลที่ได้ไปสร้างกรณีทดสอบให้ครอบคลุมการเรียกใช้งานทั้งหมด
แต่เนื่องด้วยสภาพแวดล้อมการพัฒนาจริงนั้นไม่สามารถทำได้ทั้งหมด เนื่องจากการเรียกใช้งานระหว่างกันนั้น
มีจำนวนมาก ดังนั้นงานวิจัยจึงมีแนวคิดที่จะค้นหาแนวทางในการช่วยวิเคราะห์ชุดรหัสต้นฉบับ เพื่อให้ได้มาซึ่ง
ข้อมูลการเรียกใช้งานระหว่างกัน โดยนำเสนอการเรียกใช้งานระหว่างกันนี้ในรูปแบบของแผนภาพการเรียกใช้
แล้วจึงนำข้อมูลที่ได้มาสร้างเป็นกรณีทดสอบโดยอัตโนมัติ ทั้งนี้คาดหวังว่าจะสามารถสร้างกรณีทดสอบให้
ครอบคลุมการเรียกใช้งานระหว่างกัน และลดเวลาในการสร้างกรณีทดสอบลงจากเดิมที่มีการสร้างกรณีทดสอบ
ด้วยผู้ทดสอบเอง
